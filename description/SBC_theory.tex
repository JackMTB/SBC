\section{Sideband cooling theory}
Consider a system described by the following Hamiltonian
\begin{equation}\label{freeHamm}
	\hham_0 = \frac{\hbar \omega_0}{2} \hat{\sigma}_z + \hbar \omega_{tr} \left(\hat{n} + \frac{1}{2}\right),
\end{equation}
where $\hat{\sigma_z} = \ket{e}\bra{e}-\ket{g}\bra{g}$ is the Pauli sigma-z operator and $\hat{n}$ is the number operator that acts on the eigenbasis $\{\ket{n}| n \in \mathbb{Z}^+_0\}$ as $\hat{n}\ket{n}=n\ket{n}$. The first term of equation \eqref{freeHamm} represents the simplest model of the electronic structure of an atom/ion: a "ground" and "excited" states separated in energy by $\hbar \omega_0$. The second term of equation \eqref{freeHamm} describes a quantum harmonic oscillator (QHO) characterised by a frequency $\omega_{tr}$. Therefore there are two contributions to the energy of the system, one from the electronic state of the system and one from it's "motional state" originating from the QHO.

Sideband cooling (SBC) is a technique used to reduce the average motional state of a trapped two-level system. 